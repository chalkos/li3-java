%
% Modelo LaTeX baseado no modelo A5L
% Adaptação para LaTeX por Bruno Ferreira

% Nota: usar /newpage com 1 coluna
%       usar /clearpage com 2 colunas
%       o /newpage com duas colunas escreve na próxima 
\documentclass[a5paper,twocolumn, 11pt]{article}
\usepackage[landscape]{geometry}

%Sample text
%\usepackage{lipsum}

%escrever acentos e coisas do género sem que o latex se desoriente
\usepackage[utf8]{inputenc}

%hifenização e titulos em português
\usepackage[portuges]{babel}

%para ter a informação de quantas páginas tem o documento
\usepackage{lastpage}

%importar cores predefinidas
\usepackage[usenames,dvipsnames]{xcolor}
\definecolor{DarkGray}{gray}{0.40}

%usar multirow e multicolumn
\usepackage{multirow}

%para ter imagens, depois define a directoria de imagens
\usepackage{graphicx}
\graphicspath{{./imagens/}}

%definir o cabeçalho e rodapé
\usepackage{fancyhdr}
\pagestyle{fancy}
\fancyhead[L]{
    \small{
        \textcolor{DarkGray}{
            \textbf{Uminho 2012 - LI3 --- Transitários LEI}
        }
    }
}
\fancyhead[R]{
    \small{
        \textcolor{DarkGray}{
            \textbf{Pág. \thepage\ /\pageref{LastPage}}
        }
    }
}
\fancyfoot[C]{}

%definir regras de hifenização
\hyphenation{
chaining
Hash
}

%definir comando \hyph{}, necessário para n\hyph{}dimensionais
\def\hyph{-\penalty0\hskip0pt\relax}

\begin{document}

\onecolumn
\thispagestyle{empty}
\begin{tabular}{ll}
    \multirow{7}{*}{ \includegraphics[height=90pt]{logo.jpeg} }
    &\\
    & \textcolor{DarkGray}{\Large{\textbf{Escola de Engenharia}}} \\
    &\\
    & \large{Departamento de Informática}\\
    &\\
    &\\
    & \large{Licenciatura em Engenharia Informática}\\
\end{tabular}
\begin{center}
    \Large{\textbf{Projecto de Laboratórios de Informática III}}\\
    \vspace{20pt}
    \Large{\textbf{``Transistários LEI --- Projecto 2: Parte I''}}\\
    \vspace{15pt}
    \begin{tabular}{r@{, }l}
        Bruno Ferreira&A61055\\
        Daniel Carvalho&A61008\\
    \end{tabular}
    
    \vspace{5pt}
    \emph{Grupo 42}\\\vspace{15pt}
    \large{\textbf{Braga, Maio de 2012}}
\end{center}

\newpage
\twocolumn
\tableofcontents
\newpage
\listoffigures

\newpage
\section{Resumo}
Neste relatório encontram-se explicitados dados relativos ao desempenho do uso de colecções de Java para o mesmo tipo de problema do projecto anterior realizado nesta unidade curricular. Tem-se como objectivo medir o desempenho das várias colecções aplicadas aos mesmos dados de forma a ser possível recolher informação que permita avaliar e obter conclusões concretas relativamente à comparação entre o uso das ditas colecções. O factor de medição é o tempo dispendido nas operações implementadas para as diversas colecções a avaliar. A exposição do desempenho das colecções a usar será exposto neste relatório com o suporte de gráficos e análises estatisticas contendo médias, desvios padrão, etc.
De forma a garantir a precisão dessa medição de desempenho são feitas 10 medições para várias grandezas de informação, dessa forma é possível proceder à ánalise dos dados com um maior leque de resultados.


\clearpage
\section{Introdução}
De forma a comparar o desempenho entre as estruturas de dados usadas no primeiro projecto e aquelas que podem ser usadas em Java são registados os tempos das operções que serão enunciadas de seguida tendo em conta a  estrutura de utilizadores e localidades. Para as operações são considerados diferentes patamares de número de elementos: 5000, 10000, 15000, 18000. 
As operações sobre utilizadores a efectuar são:
\begin{itemize}
    \item{1. carregar base de dados de utilizadores a partir de um ficheiro;}
    \item{2. inserir o registo de um novo utilizador;}
    \item{3. procurar um utilizador por nome e nif;}
    \item{4. percorrer (visitar) a estrutura e imprimir os dados dos utilizadores.}
\end{itemize}
As operações sobre utilizadores a efectuar são:
\begin{itemize}
    \item{1. carregar base de dados de utilizadores a partir de um ficheiro;}
    \item{2. inserir o registo de um novo utilizador;}
    \item{3. procurar um utilizador por nome e nif;}
    \item{4. percorrer (visitar) a estrutura e imprimir os dados dos utilizadores.}
\end{itemize}
As operações sobre localidades e respectivas ligações a efectuar são:
As operações sobre utilizadores a efectuar são:
\begin{itemize}
    \item{ carregar base de dados de localidades e ligações a outras localidades a
partir de um ficheiro;}
    \item{2. inserir o registo de uma nova localidade;}
    \item{3. inserir o registo de uma ligação a uma localidade;}
    \item{4. procurar as ligações de uma localidade;}
    \item{5. percorrer (visitar) a estrutura e imprimir os dados das localidades.}
\end{itemize}

No que respeita às diferentes configurações para a estrutura de dados sobre utilizadores,
são testadas e registadas informações relativas ao desempenho sobre:
\begin{itemize}
    \item{1. uma estrutura baseada em ArrayList, para fazer a ordenação pelos
dois critérios: nome e nif;}
    \item{2. uma estrutura baseada em ArrayList, para fazer a ordenação por nome
e uma estrutura auxiliar em LinkedList para a ordenação por nif;}
    \item{3. uma estrutura baseada em HashMap, para utilizadores;}
    \item{4. uma estrutura baseada em TreeMap, para utilizadores.}
\end{itemize}
No que respeita às diferentes configurações para a estrutura de dados sobre localidades,
são testadas e registadas informações relativas ao desempenho sobre:
\begin{itemize}
    \item{1. uma estrutura baseada em ArrayList, para fazer a gestão das localidades e um ArrayList para as localidades relacionadas;}
    \item{2. uma estrutura baseada em ArrayList, para fazer a gestão das localidades e um HashSet para as localidades relacionadas;}
    \item{3. uma estrutura baseada em HashMap, para localidades com um HashMap
para as localidades relacionadas;}
    \item{4. uma estrutura baseada em TreeMap, para localidades com um TreeMap
para as localidades relacionadas.}
\end{itemize}
\clearpage
\newpage
\section{Conteúdo}
\subsection{Configurações para estrutura de dados}
No seguimento deste relatório constam informações individuais e detalhadas sobre o desempenho das operações a efectuar sobre utilizadores e localidades nas diversas configurações da estrutura de dados. De forma a uniformizar os resultados, todos os testes foram efectuados na mesma máquina. As especificações técnicas dessa máquina são as seguintes:\\
Sistema Operativo: Windows 7 64bit\\
Modelo: Portátil Acer BA50-MV\\
Processador: Intel Pentium T4400 Dual-Core @ 2.2GHz\\
Caches:\\
Cache L1-D 2x32Kb\\
Cache L1-I 2x32Kb\\
Cache L2 1024Kb\\
Memória: RAM 2x2048MB 533MHz\\

Os testes sobre utilizadores são executados da seguinte forma:
Inicialmente é executada uma vez o conjunto de operações designadamente ler utilizadores do ficheiro, inserir dados de teste, pesquisar por NIF e por nome e escrever utilizadores para ficheiro. Após esta fase inicial são executados dez vezes o conjunto de operações e através dessa execução são obtidos os tempos dispendidos para tal de onde se retiram dados como média de tempo e desvio padrão.
Os testes sobre localidades são executados da seguinte forma:
Inicialmente é executada uma vez o conjunto de operações designadamente ler localidades e ligações do ficheiro, inserir dados de teste, pesquisar localidades e imprimir informações de localidades. Após esta fase inicial são executados dez vezes o conjunto de operações e através dessa execução são obtidos os tempos dispendidos para tal de onde se retiram dados como média de tempo e desvio padrão.
De notar que cada um destes testes é efectuado para diferentes patamares de número de elementos: 5000, 10000, 15000, 18000.\\

Para facilitar o uso das diferentes configurações de estrutura de dados é usado uma classe abstracta de utilizadores para podermos simplificar os métodos de cronometragem e leitura de ficheiros. Assim ao  aumentar o nível de abstração sobre a classe de utilizadores é possível proceder a uma implementação mais genérica e legível.
\subsubsection{Utilizadores: ArrayList}
\subsubsection{Utilizadores: ArrayList com LinkedList}
\subsubsection{Utilizadores: HashMap}
\subsubsection{Utilizadores: TreeMap}
\subsubsection{Localidades: ArrayList}
\subsubsection{Localidades: ArrayList com HashSet}
\subsubsection{Localidades: HashMap}
\subsubsection{Localidades: TreeSet}
\clearpage
\subsection{Estatísticas????}
\clearpage
\subsection{Estatísticas}
\subsubsection{Estatísticas gerais: Utilizadores}
\subsubsection{Estatísticas gerais: Localidades}



\clearpage
\section{Conclusão}
texto

Durante a fase de trabalhos do grupo recorreu-se ao git devido à sua capacidade de controlo de versões.

\clearpage
\onecolumn
\section{Fotos}
\begin{center}
    \begin{tabular}{ccc}
        \includegraphics[width=90pt]{bruno.png}&
        \includegraphics[width=90pt]{daniel.png}\\
        
        \small{\textbf{Bruno Ferreira}}&
        \small{\textbf{Daniel Carvalho}}\\
        \small{\textbf{A61055}}&
        \small{\textbf{A61008}}\\
    \end{tabular}
\end{center}
\end{document}